\documentclass[12pt]{article}
\usepackage[utf8]{inputenc}
\usepackage[T1]{fontenc}
\usepackage{indentfirst}
\usepackage{amsmath}
\usepackage{amssymb}
\usepackage{natbib}
\usepackage{graphicx}
\usepackage{float}
\usepackage[a4paper, margin = 2 cm]{geometry}
\usepackage{fancyhdr}
\usepackage{wrapfig}
\usepackage{hyperref}
\usepackage{mathtools}

\title{JPP opis}
\author{Dominik Wawszczak}
\date{2024-04-09}

\begin{document}
	\setlength{\parindent}{0 cm}
	
	Dominik Wawszczak \hfill Języki i Paradygmaty Programowania
	
	numer indeksu: 440014 \hfill opis języka LagerLang
	
	numer grupy: 3
    
	\bigskip
	\hrule
	\bigskip
	
	Język LagerLang bazuje na języku Latte. Zawiera cztery typy: \texttt{int},
    \texttt{bool}, \texttt{string} oraz \texttt{void}.
    
    \medskip

    Funkcje mogą zwracać wartości albo \texttt{void}. Parametry do funkcji mogą
    być podawane przez wartość albo przez referencję (przy użyciu operatora
    \texttt{\&}). Przy deklaracji funkcji określa się czy parametr jest
    podawany prze wartość, czy przez referencję.

    \medskip

    Język zawiera dodatkowo trzy operacje: \texttt{break}, \texttt{continue}
    oraz \texttt{print}.
\end{document}
